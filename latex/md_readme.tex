\subsubsection*{Intro}


\begin{DoxyItemize}
\item Succently, this is a fractal play pen.
\item It is {\itshape not} a library.
\item There is no $\ast$$\ast$\+\_\+\+A\+P\+I\+\_\+$\ast$$\ast$
\item List its most useful/innovative/noteworthy features.
\begin{DoxyItemize}
\item feature one
\item feature two
\end{DoxyItemize}
\item State its goals/what problem(s) it solves.
\begin{DoxyItemize}
\item Dust off my programming skills.
\item As much as possible work with what\textquotesingle{}s out there…
\item Continue working with $\ast$fractals$\ast$—probably the real point.
\end{DoxyItemize}
\item Key concepts.
\begin{DoxyItemize}
\item concept one
\item concept two
\end{DoxyItemize}
\item This is now and always will be {\itshape alpha}.
\item Does not include badges.
\end{DoxyItemize}

\subsubsection*{Core Technical Concepts/\+Inspiration}


\begin{DoxyItemize}
\item Why does it exist?
\item Frame your project for the potential user.
\item Compare/contrast your project with other, similar projects so the user knows how it is different from those projects.
\item Highlight the technical concepts that your project demonstrates or supports. Keep it very brief.
\end{DoxyItemize}

\subsubsection*{Getting Started/\+Requirements/\+Prerequisites/\+Dependencies}

Include any essential instructions for\+:
\begin{DoxyItemize}
\item Getting it
\item Installing It
\item Configuring It
\item Running it
\end{DoxyItemize}

\#\#\# Mini-\/\+Manual… 
\begin{DoxyCode}
void help( char c, char *Program, char *Version, char *Date ) \{
    printf( "%s v%s dated %s\(\backslash\)n", Program, Version, Date );
    if ( c == 'h' || c == '?' ) \{
        printf( "\(\backslash\)n  Options:\(\backslash\)n\(\backslash\)n" );
        printf( "  --x\_center    requires real as an argument   -x\(\backslash\)n" );
        printf( "  --y\_center    requires real as an argument   -y\(\backslash\)n" );
        printf( "  --magnify     requires real as an argument   -m\(\backslash\)n" );
        printf( "  --diameter    requires real as an argument   -d\(\backslash\)n" );
        printf( "  --iteration   requires number as an argument -i\(\backslash\)n" );
        printf( "  --width       requires number as an argument -w\(\backslash\)n" );
        printf( "  --height      requires number as an argument -l\(\backslash\)n" );
        printf( "  --file        requires string as an argument -f\(\backslash\)n" );
        printf( "  --palette     requires string as an argument -p\(\backslash\)n" );
        printf( "  --next        argument is optional number    -n\(\backslash\)n" );
        printf( "  --config      requires string as an argument -c\(\backslash\)n" );
        printf( "  --color       requires number as an argument -r\(\backslash\)n" );
        printf( "  --aa          requires string as an argument -a\(\backslash\)n" );
        printf( "  --tweak       requires number as an argument -t\(\backslash\)n" );
        printf( "  --version     no argument                    -v\(\backslash\)n" );
        printf( "  --help        no argument                    -h\(\backslash\)n" );
        printf( "\(\backslash\)n  long options ('--' prefix) are incremental till unambiguous\(\backslash\)n" );
        printf( "  short options ('-' prefix) are exact\(\backslash\)n" );
        printf( "  Hideously enough, options with optional arguments,\(\backslash\)n" );
        printf( "  take the form [-[-]]option=arg, no spaces.\(\backslash\)n" );
    \}
    exit( 0 );
\}
\end{DoxyCode}


\subsubsection*{Contributing}


\begin{DoxyItemize}
\item Chris Thomasson
\item Greg Harley
\end{DoxyItemize}

\subsubsection*{T\+O\+DO}


\begin{DoxyItemize}
\item Next steps
\item Features planned
\item Known bugs (shortlist)
\end{DoxyItemize}

\subsubsection*{Contact}


\begin{DoxyItemize}
\item \href{mailto:hsmyers@gmail.com}{\tt hsmyers@gmail.\+com}
\item \href{http://www.sdragons.org/}{\tt http\+://www.\+sdragons.\+org/}
\end{DoxyItemize}

\subsubsection*{License}


\begin{DoxyItemize}
\item see file License.\+txt 
\end{DoxyItemize}